
% !TEX encoding = UTF-8
% !TEX program = pdflatex
% !TEX spellcheck = it_IT
\documentclass[italian,a4paper]{europasscv}
\usepackage[italian]{babel}
\usepackage[backend=biber,autolang=hyphen,sorting=none,style=numeric,maxbibnames=99,doi=false,isbn=false,maxcitenames=2]{biblatex}
\usepackage{csquotes}
\usepackage{europasscv-bibliography}
\bibliography{europasscv_example}
\ecvbibhighlight{Refolli}{Francesco}{R.F.}
\ecvname{Refolli Francesco}
\ecvaddress{Piacenza, IT}
\ecvemail{francesco.refolli@gmail.com}
\ecvgithubpage{www.github.com/frefolli}
\ecvgitlabpage{www.gitlab.com/frefolli}
\ecvlinkedinpage{www.linkedin.com/in/francesco-refolli-ba630b186}
\ecvgender{Male}
\ecvnationality{Italian}
\begin{document}
\begin{europasscv}
\ecvpersonalinfo
\ecvsection{Esperienze Lavorative}
\ecvtitle{Sep 2023--Dec 2023}{Tecnico Analista Programmatore}
\ecvitem{}{Binary System SRL, Piacenza, Italia}
\ecvitem{}{Rifacimento di interfacce web, di sync per importazione dati e del sistema di calcolo delle scadenze.}
\ecvitem{}{Tecnologie: Ruby on Rails, VueJS, Python}
\ecvitem{}{\href{www.binarysystem.eu}{Vai a Sito: www.binarysystem.eu}}
\ecvsection{Pubblicazioni}
\ecvtitle{4 Aprile 2025}{ENASE Conference 2025}
\ecvitem{}{Titolo: Lessons learned from implementing a language-agnostic dependency graph parser}
\ecvitem{}{Parole chiave: Source Code, Software Analysis, Dependency Graphs}
\ecvitem{}{Categoria: Short Paper}
\ecvitem{}{\href{https://enase.scitevents.org/}{Vai a Sito: https://enase.scitevents.org/}}
\ecvsection{Manufatti}
\ecvtitle{2025}{Unconventional reinforcement learning on traffic lights with SUMO}
\ecvitem{}{Tipologia: TESI}
\ecvitem{}{Redatto in collaborazione con il prof. Giuseppe Vizzari (UnimiB) nel contesto delle attivit\`a del Centro Nazionale per la Mobilit\`a Sostenibile}
\ecvtitle{2023}{Un framework multi-linguaggio per l'identificazione delle dipendenze del codice sorgente}
\ecvitem{}{Tipologia: TESI}
\ecvitem{}{Redatto in collaborazione con il dott. Darius Sas (TXT Arcan s.r.l.)}
\ecvitem{}{\href{https://frefolli.github.io/skullian-thesis/thesis.pdf}{Vai a PDF: https://frefolli.github.io/skullian-thesis/thesis.pdf}}
\ecvsection{Istruzione e Formazione}
\ecvtitle{2023--Sep 2025}{Corso di Laurea Magistrale}
\ecvitem{}{Università degli Studi Milano Bicocca, Milano, Italia}
\ecvitem{}{Corso di laurea: Informatica [F1801Q]}
\ecvitem{}{Discipline affrontate: Modelli e Simulazioni, Ingegneria del Software (e Re-Engineering), Metodi del Calcolo Scientifico, Machine Learning, Teoria dell'Informazione, Microservizi, CI/CD, Information Retrieval}
\ecvitem{}{Anno: 2°}
\ecvitem{}{\href{unimib.it}{Vai a Sito: unimib.it}}
\ecvtitle{2020--2023}{Corso di Laurea Triennale}
\ecvitem{}{Università degli Studi Milano Bicocca, Milano, Italia}
\ecvitem{}{Corso di laurea: Informatica [E3101Q]}
\ecvitem{}{Discipline affrontate: Linguaggi di Programmazione, Reti e Sistemi Operativi, Sistemi Distribuiti, Basi di Dati, Bioinformatica, Ingegneria del Software, Statistica}
\ecvitem{}{Voto: 107}
\ecvitem{}{Tesi: Un framework multi-linguaggio per l'identificazione delle dipendenze del codice sorgente}
\ecvitem{}{\href{https://github.com/frefolli/skullian-thesis}{Vai a Tesi: https://github.com/frefolli/skullian-thesis}}
\ecvitem{}{\href{unimib.it}{Vai a Sito: unimib.it}}
\ecvtitle{2021--2021}{Progetto Cyberchallenge}
\ecvitem{}{Università degli Studi Milano Bicocca, Milano, Italia}
\ecvitem{}{\href{https://cyberchallenge.it/about}{Vai a Sito: https://cyberchallenge.it/about}}
\ecvitem{}{Allegato: Certificato}
\ecvitem{}{\includegraphics[width=12cm]{./build/CCIT\_2021\_Attestato\_partecipazione.pdf}}
\ecvtitle{2015--2020}{Diploma di Liceo Scientifico}
\ecvitem{}{Istituto di Istruzione Superiore Enrico Mattei, Fiorenzuola d’Arda, Ita}
\ecvitem{}{Voto: 75 / 100}
\ecvitem{}{\href{www.istitutomatteifiorenzuola.edu.it}{Vai a Sito: www.istitutomatteifiorenzuola.edu.it}}
\ecvsection{Competenze}

\ecvmothertongue{Italiano}
\ecvlanguageheader
\ecvlanguage{Inglese}{B2}{B2}{B2}{B2}{B2}
\ecvlanguagecertificate{\href{https://bestr.it/verify/DN0EAJ2GH7}{Bbetween Foreign Languages - English B2}}
\ecvlanguagefooter
\ecvdigitalcompetence{\ecvProficient}{\ecvProficient}{\ecvProficient}{\ecvProficient}{\ecvProficient}
\ecvblueitem{Linguaggi di Programmazione}{
\begin{ecvitemize}
\item C
\item C++
\item Rust
\item Python
\item Ruby
\item JS
\item Java
\item R
\item Lisp
\item Julia
\end{ecvitemize}
}
\ecvblueitem{Patente tipo B}{
\begin{ecvitemize}
\item Emittente: Dipartimento dei Trasporti Terrestri
\item Rilasciato: 21/01/2020
\item Scadenza: 17/07/2030
\end{ecvitemize}
}
\ecvblueitem{MOS -- Microsoft Excel 2010}{
\begin{ecvitemize}
\item Exam 77--882 MOS
\item Numero Esame: 34624028
\item Punteggio: 875
\end{ecvitemize}
}
\ecvsection{Aspirazioni Professionali}
\ecvitem{}{Nel breve termine miro a terminare gli studi. Successivamente intendo perseguire una carriera professionale negli ambiti Aerospace e Ferrovie come sviluppatore e progettista software.}
\ecvsection{Privacy}
\ecvitem{}{Autorizzo il trattamento dei miei dati personali presenti nel CV ai sensi dell’art. 13 d. lgs. 30 giugno 2003 n. 196 - "Codice in materia di protezione dei dati personali" e dell’art. 13 GDPR 679/16 - "Regolamento europeo sulla protezione dei dati persona"}
\end{europasscv}
\end{document}
